\documentclass{homeworg}
\usepackage{mathtools}
\usepackage{biblatex} %Imports biblatex package
\addbibresource{mybibliography.bib} %Import the bibliography file
\title{CS31920 Advanced Algorithms Assignment: Applying Linear Programming }
\author{Oisín Brady [oib]}

\begin{document}

\maketitle

\section{Introduction}
The problem involves a rectangular puzzle grid. Each field of the grid contains either a color or is empty. For all fields with a color there is exactly one other field that contains the same color, such that each color appears exactly twice in the grid. It is required to draw a path connecting each color pair simultaneously such that: each step in the path is a non-diagonal step towards an adjacent field, each field contains \textit{at most} one mark denoting it is part of a connection (i.e. for each field of a colored path, p, none of the fields of another colored path are in common with p). The remainder of this report will discuss the following; the translations necessary to model the puzzle into a maximum flow problem; designing a linear program to solve the maximum flow problem; the implementation of the model written in the C programming language; a conclusion which reflects on the approach used to solve this problem and whether I am happy with its outcome.

\section{Modelling}
The problem can be modelled as a maximum flow problem before being solved via linear programming.
Reducing to a maximum flow problem requires translating the introduced problem into the key components of a maximum flow problem. Namely: a flow network, a flow, a value of flow, and an objective. 
\subsection{Translating Problem to a Flow Network}
The definition of a maximum flow problem is split into two sections:
\subsubsection{Input}
An asymmetric, weighted, directed graph \(G = (V, E, c)\) with source node \(s \in V\), sink node \(t \in V\), and positive edge capacities \(c:E\rightarrow R+\). A source has no incoming edges \((v,s) \notin E \text{ for all } v \in V\). Asymmetric means \((v,w) \in E \implies (w,v) \notin E\). 
\subsubsection{Translating Problem to Max Flow Input}
Let each field in the puzzle grid be a node in graph. For each pair of fields of the same color, one is arbitrarily selected as the source node and the other as the sink node. An edge is introduced for each color for each connection between non-diagonally adjacent fields (nodes) in the puzzle grid. For example, if a puzzle has two color pairs represented by the integers 1 and 2, then for every edge not involving a source or sink node, we would introduce the same edge but for each color. I.e., the edge between (non-source/sink) adjacent nodes 3 and 4 in a puzzle grid would produce the following edges: e(3,4,1), e(3,4,2), where the first two values of e represent the nodes connected by an edge and the last value represents the color. Edges are not introduced if they would be incoming edges to a source node. Similarly, with sink nodes, edges are not introduced if they would be outbound edges from a sink node. The positive edge capacities will be introduced as bounds of the linear program that will later be described. Additionally, the linear program described later will also ensure graph asymmetry through the use of constraints. 
\subsubsection{Output}
\begin{enumerate}
    \item[(1)] A flow \(f:E\rightarrow \mathbb{R}_0^+\) that respects the capacity constraints:
        \begin{enumerate}
        	\item[] \(f(e)\leq c(e) \text{ for all } e \in E\)
        \end{enumerate}
    \item[(2)] Guarantees flow conservation at all nodes except source and sink: 
        \begin{enumerate}
        	\item[] \(\displaystyle\sum_{u \in V} f(u, v) = \sum_{w \in V} f(v, w) \text{ for all }v\in V \text{\textbackslash} \{s, t\}\)
        	\end{enumerate}
	\item[(3)] Maximises the value:
        \begin{enumerate}
        	\item[] \(v(f)=\displaystyle\sum_{v \in V}f(s, v)\)
        \end{enumerate}
\end{enumerate}
\subsubsection{Translating Problem to Max Flow Output (Using LP)}
I implemented a linear program involving the capacity constraints, flow conservation, and maximisation of flow leaving the source, along with additional constraints specific to this problem:\par
\(\text{Maximise} \displaystyle\sum_{v \in V, (s,v) \in E}x_e \text{ for all source nodes, s}\\
S.t\\\)
\begin{enumerate}
    \item[(1)] 
        \begin{enumerate}
        	\item[] \(x_e \leq 1 \text{ for all }e \in E\)
        \end{enumerate}
    \item[(2)]
        \begin{enumerate}
        	\item[] \(\displaystyle\sum_{u \in V}x_(u,v) = \displaystyle\sum_{w \in V}x_(v,w) \text{ for all } v \in V \text{\textbackslash} \{s, t\} \)\
        \end{enumerate}
    \item[(3)] 
        \begin{enumerate}
        	\item[] \(\displaystyle\sum_{u \in V}x_(u,v) \leq 1 \text{ for all } v \in V \text{\textbackslash} \{s\} \)\
        \end{enumerate}
    \item[(4)] 
        \begin{enumerate}
        	\item[] \(\displaystyle\sum_{u \in V}x_(u,v) = \displaystyle\sum_{w \in V}x_(v,w) \text{ for all } v \in V \text{\textbackslash} \{s, t\}, \text{where, } x(u,v)_{color} = x(u,w)_{color} \)\
        \end{enumerate}
    \item[(5)] 
        \begin{enumerate}
        	\item[] \(x_e \geq 0 \text{ for all } e \in E\)
    \end{enumerate}
\end{enumerate}

Where,
\begin{enumerate}
    \item[] \(X_e\) denotes the flow value of each edge in the graph, introduced in section 2.1.2.
    \item[] \(v_{color}\) denotes the what source-sink pair this edge is part of (described earlier in section 2.1.2)
\end{enumerate}

\newpage
\subsubsection{Objective function}
The objective function calls to maximise the sum of the flow values leaving all source nodes. Since all edges have a capacity <= 1 (1), and all paths are non-branching(3), a valid solution is one in which the objective value = the number of color pairs in the input puzzle grid. 

\subsubsection{Constraints}
\begin{enumerate}
    \item[(1)] Ensures that the flow value of each edge respects the capacity by not exceeding a flow value of 1. This constraint, combined with (5) constructs the bound for all edges, e, so that e = [0,1]. This means that edges will either be part of a color pair path or not (the program will later dismiss fractional values).
    \item[(2)] Ensures the law of flow conservation. This means that the total incoming flow of a node must be equivalent to its total outgoing flow. This does not apply for the source or sink node as sinks only have incoming edges and sources only have outgoing edges
    \item[(3)] Ensures that paths do not branch. I.e., by ensuring that all edge flow values are at most 1, we avoid having paths that split into separate paths or multiple paths merging into one (The implementation will disregard fractions, effectively making the edge flow value all or nothing (either 1 or 0)).
    \item[(4)] Ensures that the paths between different colors do not share nodes.
    \item[(5)] Ensures that edge values are at least 0. Used in combination with (1) to produce a bound for all edges.
\end{enumerate}

\section{Implementation}
The implementation is written in C and the linear program is constructed and solved using the GLPK library\cite{1}. My section of the program (that which was not already featured in the template code provided in the "puzzleSolver.c" file) involves the following process:\\
\begin{enumerate}
    \item[1:] Creating an LP problem object with the objective function set to maximisation
    \item[2:] Creating an adjacency matrix based of the input file (already handled in puzzleSolver.c template code) that maps all possible edges between nodes in the puzzle grid. The adjacency matrix is later used to determine what edges variables are introduced. All of these edges are given constraints depending on their involved nodes and the color pair they belong to.
    \item[3:] Creating all LP bounds (combined (1) \& (2) of section 2.1.4, implemented using glpk functions)
    \item[4:] Creating all LP constraints (See section 2.1.4) for each node (using glpk functions).
    \item[5:] Solving glpk LP via the Simplex algorithm (using a glpk function).
    \item[6:] Determining whether a "solved" optimal solution draws all paths between color pairs simultaneously. I.e., determine if the input puzzle is solvable or not.
\end{enumerate}


\section{Discussion and Conclusion}

\section{References}

\printbibliography


\end{document}
